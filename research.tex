\documentclass{article}

% Some basic packages
\usepackage[utf8]{inputenc}
\usepackage[T1]{fontenc}
\usepackage{textcomp}
\usepackage{url}
\usepackage{graphicx}
\usepackage{float}
\usepackage{booktabs}
\usepackage{enumitem}
\usepackage{array}

% Don't indent paragraphs, leave some space between them
\usepackage{parskip}

% Math stuff
\usepackage{amsmath, amsfonts, mathtools, amsthm, amssymb}

% Define Dart code styling
\usepackage{listings}
\usepackage{xcolor}
\lstdefinestyle{dartstyle}{
    language=Java,
    keywordstyle=\color{blue},
    numberstyle=\tiny\color{gray},
    commentstyle=\color{olive},
    stringstyle=\color{violet},
    basicstyle=\ttfamily\footnotesize,
    breakatwhitespace=false,         
    breaklines=true,                 
    captionpos=b,                    
    keepspaces=true,                 
    showspaces=false,                
    showstringspaces=false,
    showtabs=false,                  
    tabsize=2
}

\lstset{style=dartstyle}

\setcounter{secnumdepth}{0}
\usepackage{hyperref}

\title{Research Notebook}
\author{Joshua Markle}
\date{}
\begin{document}

\maketitle

\tableofcontents
\newpage

\section{Timesheet}

\begin{tabular}{|p{3cm}|p{3cm}|}
	\hline
	\textbf{Date} & \textbf{Time Completed} \\ \hline
	3 October 2023 & 2h 30m \\ \hline
	4 October 2023 & 1h 30m \\ \hline
	5 October 2023 & 2h \\ \hline
\end{tabular}
\newpage



\section{App Setup}

In this section, I will research different methods of app development with the goal of deciding how to build my app and what I will use to make it.

\subsection{3 October 2023 - App Frameworks}

In order to make an app, some sort of framework that provides a standard way to build and deploy applications.

\textbf{What I am looking for in a framework:}

\begin{itemize}
				\item Cross platform: Can build to both iOS and Android
				\item Established community: Pre-existing libraries and a place to go if I run into errors
\end{itemize}

In my search for suitable framworks, I have found that the two most popular ones are probably going to give me the best success in this project. These two frameworks are called React and Flutter, and dominate most of the app building options.

\begin{tabular}{| l | p{4.8cm} | p{4.8cm} |}  
\hline  
 & React Native & Flutter \\ \hline
General & Made by facebook, React uses JavaScript to make it's components. It offers a variety of UI options and more native experience & Flutter was made by Google and has it's own take on the app development process. It uses Dart to render widgets (display components) \\ \hline
Pros & 

\begin{itemize}
				\item Easy to learn and master (JavaScript)
				\item Native UI unique to iOS and Android
				\item Best for complex animations
\end{itemize}

& 

\begin{itemize}
				\item Dart is similar to other languages I know
				\item Excelent documentation
				\item Standard UI across both iOS and Android
				\item Highly performant
\end{itemize}

\\ \hline
Cons & 

\begin{itemize}
				\item Poor documentation
				\item Constant updates (many things get deprecated)
				\item Overwelhming number of options/solutions
\end{itemize}

& 

\begin{itemize}
				\item Harder to master
				\item Limited libraries by 3rd parties
				\item Large app sizes
\end{itemize}

\\ \hline
\end{tabular}  


\textbf{Sources}
\begin{itemize}
	\item \href{https://www.theseus.fi/bitstream/handle/10024/146232/thesis.pdf}{React Native vs Flutter, cross-platform mobile application frameworks (Wenhao Wu)}
	\item \href{https://www.simplilearn.com/tutorials/reactjs-tutorial/flutter-vs-react-native}{Flutter vs. React Native: Which One to Choose in 2023? (Chinmayee Deshpande)}
	\item \href{https://www.browserstack.com/guide/flutter-vs-react-native}{Flutter vs React: A Comparison}
\end{itemize}

\subsection{4 October 2023 - Project Setup}

After conducting the research on these two popular frameworks, I have decided to go with the Flutter framework ultimately because of the development process. It has better documentation that is simple and straight to the point and can make working on apps quicker and cause less suffering.

I have setup my development environment using Neovim, a popular text editor that I have been using for a while. It is highly configurable and I plan of leveraging exactally that to hasten development.

\textbf{Neovim packages installed:}

\begin{itemize}
				\item flutter-tools.nvim by akinsho
				\item vim-lsc and vim-lsc-dart by natebosch
\end{itemize}

(Packages can be found on GitHub)

Both of these packages allowed me to setup my project and provide helpful code completion. I also installed the Android operating system onto my laptop inorder for me to run a virtual phone on my computer to test out my app in realtime.

Currently, I have created a blank project and am fully preped for development.
\newpage


\subsection{5 October 2023 - Common Knowledge \& Best Practices}

Before work can be started on any large project, it is important that there is a plan to guide development and ensure that time is not wasted on extra tasks.

\textbf{Notes pertaining to building a flutter app:}

\begin{itemize}
    \item Widgets: Building blocks of UI
    \item Stateless Widgets:
    \begin{itemize}
        \item Don't store mutable state
        \item Rebuild only when parent changes
    \end{itemize}
    \item Stateful Widgets:
    \begin{itemize}
        \item Maintain mutable state
        \item Can rebuild when internal state changes
    \end{itemize}
    \item Packages \& Plugins:
    \begin{itemize}
        \item Reusable components
        \item Find at \url{https://pub.dev/flutter}
    \end{itemize}

    \item Basic Flutter App Structure
    \begin{itemize}
        \item \texttt{main.dart} as the entry point
        \item Define \texttt{main()} function
        \item Define at least one \texttt{Widget}
    \end{itemize}

    \item Creating a New App
    \begin{itemize}
        \item Use terminal or command prompt
        \item \texttt{flutter create [app\_name]}
        \item \texttt{cd [app\_name]}
        \item \texttt{flutter run} to start the app
    \end{itemize}

    \item Additional Resources
    \begin{itemize}
        \item Flutter Documentation: \url{https://flutter.dev/docs}
        \item Flutter Community: \url{https://flutter.dev/community}
    \end{itemize}
\end{itemize}
\newpage

Below is a sample project that builds necessary components for the app and centers text within the center of the screen.

\begin{lstlisting}
import 'package:flutter/material.dart';

void main() {
  runApp(MyApp());
}

class MyApp extends StatelessWidget {
  @override
  Widget build(BuildContext context) {
    return MaterialApp(
      home: Scaffold(
        appBar: AppBar(
          title: Text('Stateless Widget Example'),
        ),
        body: Center(
          child: Text('Hello, Flutter!'),
        ),
      ),
    );
  }
}
\end{lstlisting}

This code demonstrates a stateless widget and how Flutter structures their app components. Each class can contain Widget that returns a page to the user's screen when called. Stateful widgets on the other hand update in realtime on the user's screen unlike a stateless widget. This is the core of flutter, everything is a widget either stateless or stateful. 

\textbf{Sources Reviewed:}

\begin{itemize}
	\item \href{https://docs.flutter.dev/get-started/test-drive}{Test Drive (Flutter Documentation)}
	\item \href{https://www.freecodecamp.org/news/how-to-build-mobile-apps-with-flutter/}{How To Build Mobile Apps With Flutter (FreeCodeCamp)}
	\item \href{https://www.geeksforgeeks.org/flutter-tutorial/}{GeeksForGeeks Flutter Documentation}
\end{itemize}

\end{document}
